\documentclass{memoir}
\usepackage{dramatist}

\title{Jordan Peterson about God}
\newcommand{\subtitle}{Excerpt from Joe Rogan Experience 958 -- Jordan Peterson}

\author{Joe Rogan, Dr. Jordan B. Peterson}
\date{May 9, 2017}

\renewcommand{\maketitlehookb}{\centering\small{\subtitle}}

\newcommand{\qq}[1]{\emph{“#1”}}




\begin{document}
\maketitle{}


\begin{drama}
	\Character{Joe Rogan}{jr}
	\Character{Jordan Peterson}{jp}

\jrspeaks So the term \emph{intellectual}: the problem is that it identifies \emph{one} aspect and it sort of defines one aspect of behavioral thinking. Only one, it’s not at balance.

\jpspeaks What happens is like the intellect is raised to the status of highest God. That’s the right way to think. See, the God idea, here’s another way of thinking about it that’s quite cool. I learned this from Jung, and you can take it for what it’s worth: \emph{The highest ideal that a person holds, consciously or unconsciously, that’s their God.} It functions in precisely that manner. And people might say \qq{well, I don’t believe in God}. It’s like \qq{It depends on what you mean}.

\jrspeaks Right

\jpspeaks I’m not being foolish about that. We are very complicated creatures and we are run by all sorts of very strange things down there in the unconscious. The Greeks thought we were the playthings of the Gods because we serve \emph{lust}, we serve \emph{thirst}, we serve \emph{hunger}, we serve \emph{rage}, and those things all transcend us. So that’s why they were Gods. Rage -- that’s the war Gor. \qq{Why is it a God?} \qq{It exists forever.} It exists in all people. It takes them over and directs their behaviour. It’s a God. You can quibble about the details: \qq{No, it’s not a God!} \qq{Ok, fine. It’s a psychological force.}

\jrspeaks Do people get too hung up on that one word?

\jpspeaks Well, they don’t really \ldots{} We have to think about it functionally to some degree. We have to think about what that idea means. We have had that idea forever. It isn’t just some superstition. Jesus! You’ve got to be more sophisticated than that, man. This is partly what, I think, is unfortunate about the new atheists, let’s say. They don’t take the damn problem seriously. They think \qq{well, Christianity, that’s just a bunch of superstitions}. It’s like \qq{Really? No! Sorry, that’s just not deep enough, man.}

\jrspeaks So, what it really is, is the accounts of people trying to work out the issues of being human?

		\jpspeaks Let me give you another example. This is so cool. This is \emph{so cool!} It’s Nothrop Frye who is a biblical scholar at the University of Toronto. This was one of his ellucidations of the structure of the Bible. The Bible is actually a story. Which is weird because it’s a whole bunch of different books written by a whole bunch of different people, edited kind of willy-nilly over thousands of years and then assembled, you know, by a commitee. It’s a really strange book, but it has a narrative structure. And it sort of emerged as a collective decision across these thousands of years.

		So, the Old Testament, here’s the rough story of the Old Testament. Israel is sort of a middle power. And it rises to power and domination. And then a prophet arises and says \qq{Look, you gyus, you’re all successful now. You’re starting to get corrupt. You’re not paying attention to the widows and children. You’re not running your state according to the superordinate principle.} You might say \qq{The superordinate priciple doesn’t exist}. It’s like \qq{Ok, keep a-running it that way. And see what the hell happens.} That’s what the prophet says, usually at the risk of his life, he says that to the king. It’s like \qq{Fine! You don’t believe in God? You don’t believe in the superordinate priciple? \ldots{}} Let’s say that \qq{\ldots{} the superordinate ethical principle? No problem! Keep doing what you’re doing. Let’s see what happens.} What happens is Israel gets \emph{wiped out}. Then for generations it’s enslaved or its population is being destroyed and then it sort of climbs back to power. And then it gets powerful for a brief period of time, and it gets corrupt, and a prophet comes up and says: \qq{You remember that superordinate principle that you made a convenant with?You’re not paying any attention to it anymore. You better look the hell out!} And everyone ignores it and bang! So it’s order -- corruption -- chaos -- order -- corruption -- chaos. That happens six times!

There is an idea behind it. Because the state keeps arising, there is an idea which emerges out of that, that the \emph{aim is the perfect state}. That’s a utopian dream that arises out of that, let’s call it \emph{learned process}, over thousands of years. \qq{If we can only get the state perfect. Let’s say like the state of Israel, or the Russian state, the communist state. If we could only bring utopia in at the political level, all problems would be solved.} Well, what happens is there is a transition in conceptualization. And that happens with the New Testament. And the New Testament conceptualization is: \qq{Wait a minute! The state isn’t salvation! The individual is salvation! And now you say well, we’re going to just throw that out, are we? That was a hell of a discovery, man!}.

And then there’s more to it. It’s: Not only is the individual salvation, it’s the \emph{truthful} individual that’s salvation. You think of how difficult a concept like that is to develop. If there is anything less self-evident that that! You know, because you think \qq{who’s going to run the dominance hierarchy?} It’s like \qq{the biggest bloody monster with a club!} It’s like \qq{No! Turns out those are unstable!} Those societies are unstable. They don’t work, they collapse into chaos. They get corrupt. They loose sight of the superordinate principle, whatever that is. The stable solution is the individual that tells the truth. And it has taken us forever to figure that out. And that’s partly what the postmodernists are after. That’s their \emph{anti-phallogocentrism}. That’s why they skitter off and hide in their ideology. They are afraid to come out. They are afraid to be seen. They are afraid to speak because they have nothing to say.

So: We have to get sophisticated about this stuff or we are going to throw it away without understanding it. It’s unbelievable! It’s the story upon which western civilization is founded. That’s why Nietzsche said when God was dead that everything would collapse into chaos. He did not say that triumphantly. He knew what was going to happen. So did Dostoevsky. That’s why I admire those people so much. They knew what was coming.

So what I’ve been trying to do when I’ve been guided in large part by Jung\ldots{} See, Jung took Nietzsche’s problem seriously! Nietzsche said \qq{Look, we’re losing our faith. We’re losing our ability to relate to the superordinate ethical priciple.} And he actually blamed Christianity for killing itself with the sword of truth that it had produced. He said \qq{we’re going to lose this and it’s big trouble, make no mistake about it because our whole society is founded on those principles. We get rid of the animating spirit at the base of it, we’re going to lose all of it!} So, and Nietzsche thought \qq{well, we’re going to have to become superhuman to manage it}. That’s where his concept of the over-man comes from, or the superman, which the Nazis sort of pulled off and parodied I would say. Now Jung, see Jung was a student of Nietzsche. Not directly, but very much influenced by him. Jung thought that Nietzsche was wrong, that we could not create our own values. Because look: It’s so hard to create your own value. Let’s say you’re kind of an overweight guy and you decide to go to the gym, for you new year’s resolution. It’s like \qq{You don’t!} You ho twice and then you stop and it is because you cannot create your own values, right? It’s hard, you’re not your own slave. You can’t just tell yourself what to do. You have a nature. And so Jung’s idea was that we had to go back to the \emph{mythology}. We had to go back to the \emph{stories}. We had to go back into the underground unconscious chaos and lift out what we had forgotten. And that’s what he was trying to do with his psychology and he has done it very effectively. Very, very effectively. He was a revolutionary thinker, but very difficult to understand. I’ve been working with Jung’s ideas for a very long time trying to, I would say, make them more rational and articulate. No, believe me, that’s no critique because every time I go back to Jung, which I do from time to time, thinking I have kind of mastered him, I learn a bunch of stuff that I did not known. What I’ve been trying to do is to resurrect \emph{the dormant logos}, I suppose, if you have to put it that way. That’s what I’m trying to do. Mostly in men. And they are starving for it \ldots{}

\jrspeaks Why mostly in men?

\jpspeaks I don’t know. That’s just what seems to be what’s happening. About 90\% of my viewers on youtube are men. But then when I go speak publicly it’s all men. What the hell are they doing coming to hear someone speak? Men don’t do that, right? Women do that! I talk to them about truth and responsibility and their eyes light up because it’s like no one ever mentioned that before. It just boggles my mind.

\jrspeaks I’m asking because I feel like that is a gigantic theme today. That men searching for some sort of reason, for some sort of, without a better word, \emph{path}. It seems to be very, very prevalent today. In almost all walks of life men feel disenfranchised with this world they find themselves stuck in.

\jpspeaks There is a reason why superhero movies are so popular. That’s polytheism. That’s the return of polytheism. For all intents and purposes. I mean \qq{What the hell are those things?} \qq{They’re demigods.} Obviously. One of them is Thor, for God’s sake! How more obvious could it be? You might say \qq{Who is the leader of the demigods?} Because that’s the person you really want to follow. Right?

\jrspeaks Yeah.

\jpspeaks The evolutionary answer to that, as far as the Christian root went, was \emph{Christ}. But there has been lots of embodiments of that. For Mesopotamians it was \emph{Marduk}. Marduk was the savior figure. He had eyes all the way around his head. And he spoke magic words. That’s what made him different from all the other gods. He was elected by all the other gods to be their king. And then he went out and fought Tiamat who was a great dragon and made the world out of her pieces. One of his names was \emph{He-who-makes-ingenious-things-out-of-the-combat-with-Tiamat}. That’s what human beings do. They go out into the unknown, into chaos, and they make ingenious things out of it. That’s what we do. So Marduk was the founder of Mesopotamian civilization. You could think about all those tribes who came together to make Mesopotamia. They all had a god. And so then those gods went to war and out of that war of gods a meta-god emerged. That was Marduk. And Marduk is one of the sources for the figure of Christ.

That happened all over the place. You see the admirable man. Then you see ten admirable men and you think \qq{Wow, those guys have something in common!} That’s what you remember about them, you see? You remember the heroic things they’ve done because they stick in your memory, because they fit the pattern. Then you start telling the story about the heroic things that a bunch of them did and it all amalgamates together. And then you come out with your culture hero, your God! Then there is like fifty tribes, they each have their own gods. What are you going to do then? The gods go to war. Over centuries. Then they elect a new god, Marduk in the case of the Mesopotamians. He is the thing that goes out and fights the dragon of chaos and makes the world. It’s like \qq{Yeah! That’s exactly what he is!}. You do that with truth, because truth introduces you to chaos.

\jrspeaks Why do you think though it’s so much of an issue with males as opposed to females in our society?

\jpspeaks Maybe females already have enough to do. \direct{\jr{} laughs} Really, really. Maybe men have to take this on voluntarily.That's what it looks like to me. Because like you can screw around until you're fifty. You can still have a family. You've got time.And you can sit down and do nothing -- if you want. You can do it. But you shouldn't, because it's horrible to do that. And people who do it know it. It's meaningless. It's a funny thing about meaninglessness. There's no such thing. When people say their lives are meaningless, that isn't what they mean. They mean I'm in pain and anxious all the time. That's what they mean. Those are \emph{meanings,} man. You don't get \emph{neutral}, you know, \qq{I'm just sitting around, I'm not feeling anything.} It's like \qq{No! Sorry! That doesn't happen.}

\jrspeaks Right, so when you say it's \emph{do nothing}, that your life is meaningless if you're \emph{doing nothing}, what do you mean by \emph{doing nothing}.

\jpspeaks Well, by not accepting any responsibility. By not lifting a great load. By not acting out the archetype of the hero. That's what people are. That's what people are, that's what men are! If they're anything. They are mythological heroes. If they're anything!

\jrspeaks Through some path, whatever it be.

		\jpspeaks There's lots of paths. Look, there is an old medieval idea -- this is the idea of the immitation of Christ, this is something Jung elaborates on a lot. He believed, this is one of the things that he said, was that \emph{the proper goal of a Christian}, roughly speaking \emph{is to enact the meta-pattern of the Christ's life in their own} to make it their own story. What did he mean by that? Well, one of the things that characterizes mythological figure of Christ, let's say, is that he takes on the burden of mortality \emph{voluntarily}. He accepts it as a precondition of existence. And we have to do that because otherwise we get resentful. Like is hard, make no mistake about it. People's lives are tragic. If you pick a random person off the street and you ask them about their life, man, usually there's things that have happened there, \ldots{} You know, they just beggar the imagination. It's no wonder people are angry and resentful and bitter. But the way out of it is to accept it. To accept your mortality. And that helps you transcend it. That's partly what the crucifix symbol means because it was accepted voluntarily. You have to accept your death \emph{voluntarily}. That's part of the path of the hero. It's a very difficult thing to do, obviously. Oviously. What's the alternative?

\jrspeaks Yeah, obviously. You know, I think, people are constantly searching for that thing that you just described: The thing of meaning, having meaning in this life. And that meaning has a different definition for everybody. I mean your meaning might be very different that mine, or Jamie's. You kind of have to have your own path. And I think that's also one of the reasons that people are so confused, is because you're thrust, early age, into a very rigid system of education and then of jobs and then of career, structure, where you're in this place and most people don't feel like that's what they're supposed to be doing. And we feel very alienated by the very structure of society that we are embedded in.

		\jpspeaks Of course, of course. There is two primary masculine mythological figures. One is the \emph{Wise King} and the other is the \emph{king who devours his own son}. That's the “patriarchy” that the feminists are always talking about. Well, of course it's always there. So society is a destructive force, it doesn't care about you as an individual. It needs you to be a part of society. It needs you to adopt the norms and to squelch your peculiar individuality and to be a cog and to be socialized, and, you know, to hem yourself in and control yourself and not be impulsive. It's a \emph{Tyrant!} But the thing is: Society isn't \emph{only} a tyrant. That's the thing. It's like \qq{How about a little gratitude in there?} And, you know, people have a hard time with this because we like it what a thing is only one thing. But society is always two things. It's the thing that alienates you and the thing that's your \emph{Benevolent Father}. You know, it tilts. Sometimes it tilts harder towards the tyrant and that's not so good, but that's an archetypal reality.

		So, what do you have to contend with in life? (This is why these are archetypal realities, because everyone has to contend with them.) You have to contend with \emph{yourself} and the \emph{adversary that's inside you} that seems to oppose your every movement. The fact that you cannot just move forward smoothly through life without being in conflict with yourself. So there's the hero and the andversary on the individual level. And then on the social level there is the Wise King and the Tyrant. You're always going to run into that, I don't care if you're a Bantu tribesman or a, you know, New York lawyer. All those things you're going to run into. And then into the natural world you're going to run into the destructive element of nature, that's the \emph{Gorgon}. You let that thing get a glance at you and you're one, like, frozen puppy. But also there's the benevolent element of the nature, that's feminine. That's mother nature. Both of those extremes. And that's the world, that's the archetypal world. It's because it's eternal. As far as human beings are concerned those things are always there, that's our \emph{true environment}. It's not these things we see around us, they are lasting no time. These other things last forever. And that's what we're adapted to. We're adapted to the things that last forever.

		\jrspeaks Yet we go through this finite life searching for meaning.

		\jpspeaks And it's funny to note where meaning seems to locate itself. You want a meaning that justifies the suffering. It's something like that. That's a transcendent meaning. It's like \qq{This is hard, but it's worth it!}. Ok, so what do you do? Pick something worth it! Right? That's partly what I try to get people to do with that Future Authoring Program. To say \qq{ok, look, here's a place to start. You guide your miserable self. Right now.} It's like \qq{Three to five years out, imagine what your life could be like if you had what you would give yourself if you were taking care of yourself. What would life be like? Just come up with an idea even about that!} And so then people do that. And then they write out a plan to attain it. And then the college kids are like 30\% more likely to stay in university if they do that, especially if they're men. Because men need a \emph{purpose}. I think women \emph{have} a purpose. Only now they have \emph{two} purposes, they're going to have a family, that's a major purpose, man. Like \qq{Just give birth!} That's no joke. And then you're devoted to something for like twenty years. You've got your adventure right there!

		\jrspeaks Yeah, but a lot of woment find great offence in someone saying that, especially a man saying that. Mansplaining that a women's purpose is to breed, right? I mean isn't that a giant issue that a lot of women have?

		\jpspeaks Well I didn't say that was the only purpose.

		\jrspeaks I know, you didn't.

		\jpspeaks Oh yeah, people have an issue with it, but it's like \qq{Grow up!} You know, if you're a sophisticated person, as far as I'm concerned, how many important things are there in life? Well, one of them is family. It's as simple as that. Now, you might say \qq{Well, family isn't the end-all solution.} It's like \qq{Yeah, well, thanks for pointing that out}, you know? I've dealt with plenty of pathological families, but it's a huge part of life. You have a mother, a father, you have children, it places you in the world. Look, there is a reason societies worship the virgin mother and the child. It's because that societies that don't \emph{die}.

		And so people say \qq{Well, you know, that relationship between mother and a child isn't the only thing.} \qq{Ok, fine! It's still a sacred thing!} And you miss it, you miss it! If you're female and you miss that at your preril! Now that doen't mean there aren't women who shouldn't miss it. Because maybe they have another purpose that transcends that. But \emph{that's rare.} It's very, very rare. And I would caution any women listening, if they're young not to be deluded into the idea that their career will be of such high quality that it self-evidently trumps having a family. You have to have a hell of a career before that's the case. So.

		\jrspeaks Don't you think that's unique to the individual, though, that some people just, they'll be more satisfied, and, I mean, depending what they are doing artistically, or creatively or whatever it is.

		\jpspeaks Sure, yeah. You can't make rules for the exceptional. You know, they do what they're going to do. Maybe those are open people, they have genius level IQ, like, they're \emph{spectacular} in some manner. And so there is a reason they're going to step outside the norm. They are shape shifters. No problem. There is always going to be people like that and we need them.

		\jrspeaks And how they know when they are that?

		\jpspeaks Well telling the truth is a good start. Because then you don't fool yourself about who you are. One of the things I try to think through is why you should tell the truth. It's not self-evident, man, a smart kid\ldots{} The smarter the kid the earlier they learn to lie. Lying is very powerful because you can manipulate the world with your language and then you can get \emph{what you want}. Lots of times. Or escape form things that you don't want, so why not lie all the time? I think the reason is\ldots There is a bunch of reasons, but one of them is that you can't trust yourself if you lie. There is going to be times in your life when you have no one to turn to -- except you. So if you have stuffed youself full of lies then you're going to be in a crisis one day and you're going to have to make a decision and you're going to decide wrong. And you're going to be in real trouble. Because you won't have the clarity of mind necessary to make the proper judgement. Because you've filled your imagination and your perception with \ldots{} with rubbish.

		If you relly thing that throug, you see, there is this old idea in the Old Testament that \emph{fear of God is the beginning of wisdom} and I can understand what that means. One of the things we do with the Future Authoring Program: We offer people a little heaven. It's like \qq{Ok, construct your ideal! Aim at it! Come up with a plan! You've got to modify the plan? No problem. You're going to do a bad job of it? No problem. Just do it!} Ok, so now you've got a goal. Now your approach systems, technically speaking, the positive emotions systems motivate you. Now you're engaged because they are engaged in relationship to a goal. The more transcendent the goal the more are they engaged. But that's not good enough! It's great to run towards something you like but it's even better to run away from something that terrifies you. So then we ask people \qq{Ok, so here, think about really carefully:} Take all your faults and your inadequcies, and your hatred for life, and all of that, and then imagine that gets the upper hand. And then think about when you could be in three to five years. Everyone knows, hey? Some people know they'd be a street person, some people know they'd be an alcoholic, some people know they'd be a prostitute, or a drug addict. Like everybody has got their own little \emph{hell} they could descend into. With fair degree of repetity and a fair degree of enjoyment. People \emph{know} that. So I say \qq{Delineate that out, too, so you know where you're headed when you fall off the path.} And so then you're running away and running towards, it's like \qq{Yes!}. That's heaven and hell. And you need it. And they are \emph{real!} They are as real as anything that you can\ldots{} It depends on what you mean by real, I suppose, but\ldots{} They are as real as you make them, how about that? And people can make hell pretty real.t 

\end{drama}
\end{document}
