\documentclass{memoir}
\usepackage{dramatist}

\title{Jordan Peterson about God}
\newcommand{\subtitle}{Excerpt from Joe Rogan Experience 958 -- Jordan Peterson}

\author{Joe Rogan, Dr. Jordan B. Peterson}
\date{May 9, 2017}

\renewcommand{\maketitlehookb}{\centering\small{\subtitle}}

\newcommand{\qq}[1]{\emph{“#1”}}




\begin{document}
\maketitle{}


\begin{drama}
	\Character{Joe Rogan}{jr}
	\Character{Jordan Peterson}{jp}

\jrspeaks So the term \emph{intellectual}: the problem is that it identifies \emph{one} aspect and it sort of defines one aspect of behavioral thinking. Only one, it’s not at balance.

\jpspeaks What happens is like the intellect is raised to the status of highest God. That’s the right way to think. See, the God idea, here’s another way of thinking about it that’s quite cool. I learned this from Jung, and you can take it for what it’s worth: \emph{The highest ideal that a person holds, consciously or unconsciously, that’s their God.} It functions in precisely that manner. And people might say \qq{well, I don’t believe in God}. It’s like \qq{It depends on what you mean}.

\jrspeaks Right

\jpspeaks I’m not being foolish about that. We are very complicated creatures and we are run by all sorts of very strange things down there in the unconscious. The Greeks thought we were the playthings of the Gods because we serve \emph{lust}, we serve \emph{thirst}, we serve \emph{hunger}, we serve \emph{rage}, and those things all transcend us. So that’s why they were Gods. Rage -- that’s the war Gor. \qq{Why is it a God?} \qq{It exists forever.} It exists in all people it takes them over and directs their behaviour. It’s a God. You can quibble about the details: \qq{No, it’s not a God!} \qq{Ok, fine. It’s a psychological force.}

\jrspeaks Do people get too hung up on that one word?

\jpspeaks Well, they don’t really \ldots{} We have to think about it functionally to some degree. We have to think about what that idea means. We have had that idea forever. It isn’t just some superstition. Jesus! You’ve got to be more sophisticated than that, man. This is partly what, I think, is unfortunate about the new atheists, let’s say. They don’t take the damn problem seriously. They think \qq{well, Christianity, that’s just a bunch of superstitions}. It’s like \qq{Really? No! Sorry, that’s just not deep enough, man.}

\jrspeaks So, what it really is, is the accounts of people trying to work out the issues of being human?

\jpspeaks Let me give you another example. This is so cool. This is \emph{so} cool! It’s Nothrop Frye who is a biblical scholar at the University of Toronto. This was one of his ellucidations of the structure of the Bible. The Bible is actually a story. Which is weird because it’s a whole bunch of different books written by a whole bunch of different people, edited kind of willy-nilly over thousands of years and then assembled, you know, by a commitee. It’s a really strange book, but it has a narrative structure. And it sort of emerged as a collective decision across these thousands of years.

		So, the Old Testament, here’s the rough story of the Old Testament. Israel is sort of a middle power. And it rises to power and domination. And then a prophet arises and says \qq{Look, you gyus, you’re all successful now. You’re starting to get corrupt. You’re not paying attention to the widows and children. You’re not running your state according to the superordinate principle.} You might say \qq{The superordinate priciple doesn’t exist}. It’s like \qq{Ok, keep a-running it that way. And see what the hell happens}. That’s what the prophet says, usually at the risk of his life, he says that to the king. It’s like \qq{Fine! You don’t believe in God? You don’t believe in the superordinate priciple? \ldots{}} Let’s say that \qq{\ldots{} the superordinate ethical principle? No problem! Keep doing what you’re doing Let’s see what happens.} What happens is Israel gets \emph{wiped out}. Then for generations it’s enslaved or its population is being destroyed and then it sort of climbs back to power. And then it gets powerful for a brief period of time, and it gets corrupt, and a prophet comes up and says: \qq{You remember that superordinate principle that you made a convenant with?You’re not paying any attention to it anymore. You better look the hell out!} And everyone ignores it and bang! So it’s order -- corruption -- chaos -- order -- corruption -- chaos. That happens six times!

There is an idea behind it. Because the state keeps arising, there is an idea which emerges out of that, that the \emph{aim is the perfect state}. That’s a utopian dream that arises out of that, let’s call it \emph{learned process}, over thousands of years. \qq{If we can only get the state perfect. Let’s say like the state of Israel, or the Russian state, the communist state. If we could only bring utopia in at the political level, all problems would be solved.} Well, what happens is there is a transition in conceptualization. And that happens with the New Testament. And the New Testament conceptualization is: \qq{Wait a minute! The state isn’t salvation! The individual is salvation! And now you say well, we’re going to just throw that out, are we? That was a hell of a discovery, man!}.

And then there’s more to it. It’s: Not only is the individual salvation, it’s the \emph{truthful} individual that’s salvation. You think of how difficult a concept like that is to develop. If there is anything less self-evident that that! You know, because you think \qq{who’s going to run the dominance hierarchy?} It’s like \qq{the biggest bloody monster with a club!} It’s like \qq{No! Turns out those are unstable!} Those societies are unstable. They don’t work, they collapse into chaos. They get corrupt. They loose sight of the superordinate principle, whatever that is. The stable solution is the individual that tells the truth. And it has taken us forever to figure that out. And that’s partly what the postmodernists are after. That’s their \emph{anti-phallogocentrism}. That’s why they skitter off and hide in their ideology. They are afraid to come out. They are afraid to be seen. They are afraid to speak because they have nothing to say.

So: We have to get sophisticated about this stuff or we are going to throw it away without understanding it. It’s unbelievable! It’s the story upon which western civilization is founded. That’s why Nietzsche said when God was dead that everything would collapse into chaos. He did not say that triumphantly. He knew what was going to happen. So did Dostoevsky. That’s why I admire those people so much. They knew what was coming.

So what I’ve been trying to do when I’ve been guided in large part by Jung\ldots{} See, Jung took Nietzsche’s problem seriously! Nietzsche said \qq{Look, we’re losing our faith. We’re losing our ability to relate to the superordinate ethical priciple.} And he actually blamed Christianity for killing itself with the sword of truth that it had produced. He said \qq{we’re going to lose this and it’s big trouble, make no mistake about it because our whole society is founded on those principles. We get rid of the animating spirit at the base of it, we’re going to lose all of it!} So, and Nietzsche thought \qq{well, we’re going to have to become superhuman to manage it}. That’s where his concept of the over-man comes from, or the superman, which the Nazis sort of pulled off and parodied I would say. Now Jung, see Jung was a student of Nietzsche. Not directly, but very much influenced by him. Jung thought that Nietzsche was wrong, that we could not create our own values. Because look: It’s so hard to create your own value. Let’s say you’re kind of an overweight guy and you decide to go to the gym, for you new year’s resolution. It’s like \qq{You don’t!} You ho twice and then you stop and it is because you cannot create your own values, right? It’s hard, you’re not your own slave. You can’t just tell yourself what to do. You have a nature. And so Jung’s idea was that we had to go back to the \emph{mythology}. We had to go back to the \emph{stories}. We had to go back into the underground unconscious chaos and lift out what we had forgotten. And that’s what he was trying to do with his psychology and he has done it very effectively. Very, very effectively. He was a revolutionary thinker, but very difficult to understand. I’ve been working with Jung’s ideas for a very long time trying to, I would say, make them more rational and articulate. No, believe me, that’s no critique because every time I go back to Jung, which I do from time to time, thinking I have kind of mastered him, I learn a bunch of stuff that I did not known. What I’ve been trying to do is to resurrect \emph{the dormant logos}, I suppose, if you have to put it that way. That’s what I’m trying to do. Mostly in men. And they are starving for it \ldots{}

\jrspeaks Why mostly in men?

\jpspeaks I don’t know. That’s just what seems to be what’s happening. About nintey percent of my viewers on youtube are men. But then when I go speak publicly it’s all men. What the hell are they doing coming to hear someone speak? Men don’t do that, right? Women do that! I talk to them about truth and responsibility and their eyes light up because it’s like no one ever mentioned that before. It just boggles my mind.

\jrspeaks I’m asking because I feel like that is a gigantic theme today. That men searching for some sort of reason, for some sort of, without a better word, \emph{path}. It seems to be very, very prevalent today. In almost all walks of life men feel disenfranchised with this world they find themselves stuck in.

\jpspeaks There is a reason why superhero movies are so popular. That’s polytheism. That’s the return of polytheism. For all intents and purposes. I mean \qq{What the hell are those things?} \qq{They’re demigods.} Obviously. One of them is Thor, for God’s sake! How more obvious could it be? You might say \qq{Who is the leader of the demigods?} because that’s the person you really want to follow. Right?

\jrspeaks Yeah.

\jpspeaks The evolutionary answer to that, as far as the Christian root went, was \emph{Christ}. But there has been lots of embodiments of that. For Mesopotamians it was \emph{Marduk}. Marduk was the savior figure. He had eyes all the way around his head. And he spoke magic words. That’s what made him different from all the other gods. He was elected by all the other gods to be their King. And then he went out and fought Tiamat who was a great dragon and made the world out of her pieces. One of his names was \emph{He-who-makes-ingenious-things-out-of-the-combat-with-Tiamat}. That’s what human beings do. They go out into the unknown, into chaos, and they make ingenious things out of it. That’s what we do. So Marduk was the founder of Mesopotamian civilization. You could think about all those tribes who came together to make Mesopotamia. They all had a god. And so then those gods went to war and out of that war of gods a meta-god emerged. That was Marduk. And Marduk is one of the sources for the figure of Christ.

That happened all over the place. You see the admirable man. Then you see ten admirable men and you think \qq{Wow, those guys have something in common!} That’s what you remember about them, you see? You remember the heroic things they’ve done because they stick in your memory, because they fit the pattern. Then you start telling the story about the heroic things that a bunch of them did and it all amalgamates together. And then you come out with your culture hero, your God! Then there is like fifty tribes, they each have their own gods. What are you going to do then? The gods go to war. Over centuries. Then they elect a new god, Marduk in the case of the Mesopotamians. He is the thing that goes out and fights the dragon of chaos and makes the world. It’s like \qq{Yeah! That’s exactly what he is!}. You do that with truth, because truth introduces you to chaos.

\jrspeaks Why do you think though it’s so much of an issue with males as opposed to females in our society?


\end{drama}
\end{document}
